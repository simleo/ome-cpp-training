\documentclass{beamer}

\newcommand{\cmd}[1]{\textbf{\texttt{#1}}}
\newcommand{\pkg}[1]{\texttt{#1}}
\newcommand{\env}[1]{\texttt{#1}}
\newcommand{\opt}[1]{\textsl{#1}}

\usepackage{beamerthemesplit}
\beamertemplatenavigationsymbolsempty
\setbeamertemplate{footline}[frame number]{}

\usepackage{graphicx}
\usepackage{color}
\usepackage{listings}
\usepackage{verbatim}
\usepackage{setspace}
\usepackage{url}
\usepackage{hyperref}

\usepackage{fontspec}
\setmonofont{Latin Modern Mono}
\setsansfont{TeX Gyre Heros}

\lstset{breakatwhitespace=true,
language=C++,
basicstyle=\footnotesize\ttfamily,
keywordstyle=\color{blue}\ttfamily,
stringstyle=\color{red}\ttfamily,
commentstyle=\color{brown}\ttfamily,
morecomment=[l][\color{magenta}]{\#}keepspaces=true,
breaklines=true,
tabsize=3,
showstringspaces=false,
extendedchars=true,
frame=single,
escapeinside=§§
}

\newcommand*{\vcenteredhbox}[1]{\begingroup
\setbox0=\hbox{#1}\parbox{\wd0}{\box0}\endgroup}

\usepackage{tikz}
\usetikzlibrary{matrix}
\usepackage{ragged2e}

\tikzstyle{head} = [rectangle, rounded corners, minimum width=3cm, minimum height=1cm,text centered, draw=black, fill=red!30]
\tikzstyle{commit} = [rectangle, minimum width=1cm, minimum height=1cm, text centered, draw=black, fill=orange!30]
\tikzstyle{repo} = [rectangle, minimum width=2cm, minimum height=1cm, text width=2cm, text centered, draw=black, fill=orange!30]

\tikzstyle{arrow} = [thick,->,>=stealth]
\tikzstyle{darrow} = [thick,<->,>=stealth]

\newcommand*{\abox}[1]{\framebox{\hbox to 0.5cm{\texttt\{{\\hss#1\hss}}}}

\title{Using the \texttt{git} version control system}
\author{Roger Leigh}
\date{22\textsuperscript{nd} April 2013}

\begin{document}

\begin{frame}[plain,fragile]
  \titlepage

  \begin{center}
  \textsl{I'm an egotistical bastard, and I name all my projects after myself. First `Linux', now `Git'}\hspace{2cm}---Linus Torvalds
  \end{center}
\end{frame}

\section[]{Overview}
\frame{
\frametitle{Overview}
\tableofcontents
}

\begin{frame}[fragile]
  \frametitle{Revisions}

  \begin{center}
    \begin{tikzpicture}[y=-1cm]

      \onslide<1->{
        \node (rev1) [commit] {Rev 1};
        \node (rev2) [commit, right of=rev1, xshift=2cm] {Rev 2};
      }
      \onslide<2->{
        \node (rev3) [commit, right of=rev2, xshift=2cm] {Rev 3};
        \node (rev4) [commit, right of=rev3, xshift=2cm] {Rev 4};
      }
      \onslide<1-2>{
        \draw [arrow] (rev1) -- (rev2);
      }
      \onslide<2>{
      \draw [arrow] (rev2) -- (rev3);
      \draw [arrow] (rev3) -- (rev4);
      }
      \onslide<3->{
        \path[arrow] (rev1) edge [bend left=45] node[anchor=south] {$\Delta_{1\rightarrow{}2}$} (rev2);
        \path[arrow] (rev2) edge [bend left=45] node[anchor=south] {$\Delta_{2\rightarrow{}3}$} (rev3);
        \path[arrow] (rev3) edge [bend left=45] node[anchor=south] {$\Delta_{3\rightarrow{}4}$} (rev4);
      }
      \onslide<4->{
        \path[arrow] (rev2) edge [bend left=45] node[anchor=north] {$-\Delta_{1\rightarrow{}2}$} (rev1);
        \path[arrow] (rev3) edge [bend left=45] node[anchor=north] {$-\Delta_{2\rightarrow{}3}$} (rev2);
        \path[arrow] (rev4) edge [bend left=45] node[anchor=north] {$-\Delta_{3\rightarrow{}4}$} (rev3);
      }
    \end{tikzpicture}
  \end{center}
\end{frame}

\begin{frame}[fragile]
  \frametitle{Computing and applying deltas (diffs)}
  \begin{description}
  \item[\texttt{diff}] Compute the difference between two revisions (two collections of files)
    \item[\texttt{patch}] Transform one revision to another by applying a diff
    \end{description}
\end{frame}

\begin{frame}[fragile]
  \frametitle{Branches}

  \begin{center}
    \begin{tikzpicture}[y=-1cm]

      \onslide<1->{
        \node (rev1) [commit] {1};
        \node (rev2) [commit, right of=rev1, xshift=1cm] {2};
        \draw [arrow] (rev1) -- (rev2);
      }
      \onslide<2->{
        \node (rev3) [commit, right of=rev2, xshift=1cm] {3};
        \node (rev11) [commit, below of=rev3, yshift=-2cm] {f1};
        \draw [arrow] (rev2) -- (rev3);
        \path[arrow] (rev2) edge [bend right=20] node[anchor=north east] {branch} (rev11);
      }
      \onslide<3->{
        \node (rev4) [commit, right of=rev3, xshift=1cm] {4};
        \node (rev5) [commit, right of=rev4, xshift=1cm] {5};
        \node (rev12) [commit, right of=rev11, xshift=2cm] {f2};

        \draw [arrow] (rev3) -- (rev4);
        \draw [arrow] (rev4) -- (rev5);
        \draw [arrow] (rev11) -- (rev12);

      }
      \onslide<4->{
        \node (rev6) [commit, right of=rev5, xshift=1cm] {6};
        \draw [arrow] (rev5) -- (rev6);
        \path[arrow] (rev12) edge [bend right=20] node[anchor=north west] {merge} (rev6);
      }
    \end{tikzpicture}
  \end{center}
\end{frame}

\begin{frame}[fragile]
  \frametitle{repository relationships (traditional client-server)}

  \begin{center}
    \begin{tikzpicture}

      \onslide<1->{
        \node (remote) [repo] {Remote repository};
      }
      \onslide<2->{
        \node (local2) [repo, below of=remote, yshift=-2cm] {Local checkout};
        \node (local1) [repo, right of=local2, xshift=2cm] {Local checkout};
        \node (local3) [repo, left of=local2, xshift=-2cm] {Local checkout};
        \draw [darrow] (remote) -- (local1);
        \draw [darrow] (remote) -- (local2);
        \draw [darrow] (remote) -- (local3);
      }
    \end{tikzpicture}
  \end{center}
\end{frame}

\begin{frame}[fragile]
  \frametitle{repository relationships (git/distributed)}

  \begin{center}
    \begin{tikzpicture}

      \onslide<1->{
        \node (local) [repo] {Local repository};
      }
      \onslide<2->{
        \node (workdir) [repo, left of=local, xshift=-2cm] {Working directory};
        \draw [arrow] (local) -- (workdir);
      }
      \onslide<3->{
        \node (remote2) [repo, right of=local, xshift=2cm] {Remote repository};
        \node (remote1) [repo, above of=remote2, yshift=1cm] {Remote repository};
        \node (remote3) [repo, below of=remote2, yshift=-1cm] {Remote repository};
        \draw [darrow] (local) -- (remote1);
        \draw [arrow] (remote2) -- (local);
        \draw [arrow] (remote3) -- (local);
      }
      \onslide<4->{
        \node (index) [repo, below right of=workdir, yshift=-2cm] {Index (staging)};
        \draw [darrow] (workdir) -- (index);
      }
      \onslide<5->{
        \node (stash) [repo, left of=index, xshift=-2cm] {Stash};
        \draw [darrow] (workdir) -- (stash);
      }
    \end{tikzpicture}
  \end{center}
\end{frame}

\begin{frame}[fragile]
  \frametitle{Object store types}

  \begin{description}
  \item[blob] Raw file content; referenced by SHA1 hash
  \item[tree] Directory, containing blobs and nested trees; referenced by SHA1 hash
  \item[commit] Change set, containing tree hash, parent hash, author+committer, description; referenced by SHA1 hash
  \item[tag] Label for a commit, containing commit hash and description; referenced by name
  \item[ref] Branch name, pointing to a commit hash; referenced by name
  \end{description}
\end{frame}

\section[]{Acknowledgements}

\frame{
  \frametitle{Acknowledgements}
  \begin{center}
    \vcenteredhbox{\includegraphics[width=0.25\textwidth]{ome}} \hfill
    \vcenteredhbox{\includegraphics[width=0.2\textwidth]{dundee}}\hfill
    \vcenteredhbox{\includegraphics[width=0.5\textwidth]{wellcome}}
  \end{center}
}

\end{document}
