\documentclass{beamer}

\newcommand{\cmd}[1]{\textbf{\texttt{#1}}}
\newcommand{\pkg}[1]{\texttt{#1}}
\newcommand{\env}[1]{\texttt{#1}}
\newcommand{\opt}[1]{\textsl{#1}}

\usepackage{beamerthemesplit}
\beamertemplatenavigationsymbolsempty
\setbeamertemplate{footline}[frame number]{}

\usepackage{graphicx}
\usepackage{color}
\usepackage{listings}
\usepackage{verbatim}
\usepackage{setspace}
\usepackage{url}
\usepackage{hyperref}

\usepackage{fontspec}
\setmonofont{Latin Modern Mono}
\setsansfont{TeX Gyre Heros}

\lstset{breakatwhitespace=true,
language=C++,
basicstyle=\footnotesize\ttfamily,
keywordstyle=\color{blue}\ttfamily,
stringstyle=\color{red}\ttfamily,
commentstyle=\color{brown}\ttfamily,
morecomment=[l][\color{magenta}]{\#}keepspaces=true,
breaklines=true,
tabsize=3,
showstringspaces=false,
extendedchars=true,
frame=single,
escapeinside=§§
}

\newcommand*{\vcenteredhbox}[1]{\begingroup
\setbox0=\hbox{#1}\parbox{\wd0}{\box0}\endgroup}


\title{C++ Build Environment Setup for Building Bio-Formats and OmeroCpp}
\author{Roger Leigh}
\date{Wednesday 3\textsuperscript{rd} September 2014\\University of Dundee}

\begin{document}

\begin{frame}[plain]
  \titlepage
  \begin{center}
    \vcenteredhbox{\includegraphics[width=0.25\textwidth]{ome}} \hfill
    \vcenteredhbox{\includegraphics[width=0.2\textwidth]{dundee}}\hfill
    \vcenteredhbox{\includegraphics[width=0.25\textwidth]{wellcome}}
  \end{center}
\end{frame}

\section[]{Overview}
\frame{
\frametitle{Overview}
\tableofcontents
}

\section{Installation of prerequisites}
\subsection{Compiler and toolchain}
\begin{frame}
  \frametitle{Default compilers}
  \begin{itemize}
  \item FreeBSD: LLVM/\cmd{clang++} or GCC/\cmd{g++}
  \item Linux: GCC/\cmd{g++} and GNU Binutils/\cmd{ld}
  \item MacOS X: XCode (custom LLVM/\cmd{clang++})
  \item Windows: Visual Studio or Visual Studio Express (MSVC/\cmd{cl})
  \end{itemize}

  Earlier versions of MacOS X used GCC 4.2.
\end{frame}

\subsection{Package installation}
\begin{frame}
  \frametitle{Package managers}
  \begin{itemize}
  \item FreeBSD: Ports (e.g. \cmd{pkg}, \cmd{portmaster})
  \item Linux: Distribution package manager (e.g. \cmd{apt-get} or \cmd{yum})
  \item MacOS X: homebrew (\cmd{brew})
  \item Windows: Yeah, right.  You need to manually download all the
    tools and then compile all the libraries by hand for your specific
    version of Visual Studio.  (Microsoft love to make development for
    their platform easy and painless.  Not!)
  \end{itemize}

  On the next few pages, the needed packages for each platform will be
  detailed.  This includes all packages needed for Bio-Formats and
  OMERO including unit testing and API documentation generation; you
  might not need them all but it doesn't hurt to have them all.
\end{frame}

\begin{frame}
  \frametitle{FreeBSD packages}
  \scriptsize
  Run \cmd{pkg install} to install:
  \begin{columns}
    \begin{column}{.5\linewidth}
      \begin{itemize}
      \item[] \pkg{devel/apache-ant}
      \item[] \pkg{devel/boost-all}
      \item[] \pkg{devel/binutils}
      \item[] \pkg{devel/cmake}
      \item[] \pkg{devel/doxygen}
      \item[] \pkg{devel/git}
      \item[] \pkg{devel/googletest}
      \item[] \pkg{devel/ice}
      \item[] \pkg{graphics/graphviz}
      \item[] \pkg{graphics/tiff}
      \end{itemize}
      \end{column}
    \begin{column}{.5\linewidth}
    \begin{itemize}
      \item[] \pkg{java/openjdk7}
      \item[] \pkg{java/junit}
      \item[] \pkg{lang/clang33}
      \item[] \pkg{lang/python}
      \item[] \pkg{lang/python27}
      \item[] \pkg{print/texlive-full}
      \item[] \pkg{science/hdf5}
      \item[] \pkg{textproc/py-genshi}
      \item[] \pkg{textproc/py-sphinx}
      \item[] \pkg{textproc/xerces-c3}
    \end{itemize}
    \end{column}
  \end{columns}

  \bigskip
  Add \path{/usr/local/bin} before \path{/usr/bin} in the \env{PATH}
  so that the newer GNU \cmd{ld} is used.
\end{frame}

\begin{frame}
  \frametitle{Debian and Ubuntu packages}
  \scriptsize
  Run \cmd{apt-get install} to install:
  \begin{columns}
    \begin{column}{.5\linewidth}
      \begin{itemize}
      \item[] \pkg{ant ant-contrib ant-optional}
      \item[] \pkg{build-essential}
      \item[] \pkg{cmake}
      \item[] \pkg{doxygen}
      \item[] \pkg{git}
      \item[] \pkg{graphviz}
      \item[] \pkg{junit4}
      \item[] \pkg{libboost-all-dev}
      \item[] \pkg{libgtest-dev}
      \item[] \pkg{libtiff5-dev}
      \item[] \pkg{libxerces-c-dev}
      \end{itemize}
    \end{column}
    \begin{column}{.5\linewidth}
      \begin{itemize}
      \item[] \pkg{libzeroc-ice35-dev ice35-services ice35-slice ice35-translators icebox python-zeroc-ice}
      \item[] \pkg{openjdk-7-jdk openjdk-7-jre}
      \item[] \pkg{python python2.7}
      \item[] \pkg{texlive-full}
      \item[] \pkg{libhdf5-dev}
      \item[] \pkg{python-genshi} (or use \cmd{pip})
      \item[] \pkg{python-sphinx} (or use \cmd{pip})
      \end{itemize}
    \end{column}
  \end{columns}
\end{frame}

\begin{frame}
  \frametitle{CentOS and RHEL packages}
  \scriptsize
  Run \cmd{yum groupinstall "Development Tools"}
  \bigskip

  Run \cmd{yum install} to install:
  \begin{columns}
    \begin{column}{.5\linewidth}
      \begin{itemize}
      \item[] \pkg{boost-devel}
      \item[] \pkg{cmake}
      \item[] \pkg{doxygen}
      \item[] \pkg{git}
      \item[] \pkg{graphviz}
      \item[] \pkg{gtest-devel}
      \item[] \pkg{hdf5-devel}
      \end{itemize}
    \end{column}
    \begin{column}{.5\linewidth}
      \begin{itemize}
      \item[] \pkg{java-1.7.0-openjdk}
      \item[] \pkg{junit4}
      \item[] \pkg{libtiff-devel}
      \item[] \pkg{python-genshi} (or use \cmd{pip})
      \item[] \pkg{python}
      \item[] \pkg{texlive-full}
      \item[] \pkg{xerces-c-devel}
      \end{itemize}
    \end{column}
  \end{columns}

  \bigskip
  Install the following by hand:

  \begin{itemize}
  \item Ant
  \item JUnit
  \item Ice (RPMs available)
  \item \TeX{}Live (via \cmd{install-tl})
  \item \pkg{sphinx} (via \cmd{pip})
\end{itemize}
\end{frame}

\begin{frame}
  \frametitle{MacOS X homebrew packages}
  \scriptsize
  Install XCode and its command line tools
  \bigskip

  Run \cmd{brew install} to install:
  \begin{columns}
    \begin{column}{.5\linewidth}
      \begin{itemize}
      \item[] \pkg{ant}
      \item[] \pkg{boost}
      \item[] \pkg{cmake}
      \item[] \pkg{doxygen}
      \item[] \pkg{git}
      \item[] \pkg{graphviz}
      \end{itemize}
    \end{column}
    \begin{column}{.5\linewidth}
      \begin{itemize}
      \item[] \pkg{hdf5}
      \item[] \pkg{ice}
      \item[] \pkg{libtiff}
      \item[] \pkg{python}
      \item[] \pkg{xerces-c}
      \end{itemize}
    \end{column}
  \end{columns}
\bigskip
Install the following by hand:
\begin{itemize}
\item Google Test (\pkg{gtest}) from zip or subversion
\item Java (JDK 1.7 from Oracle)
\item \TeX{}Live (via \cmd{install-tl} or Mac\TeX{})
\item \pkg{sphinx} (via \cmd{pip})
\end{itemize}
\end{frame}

\begin{frame}
  \frametitle{Windows installation (packages)}
  Install the following by hand:
\begin{itemize}
\item Ant
\item CMake
\item Doxygen and Graphviz
\item Git (msysgit)
\item Ice (latest ZeroC installer or our 3.5.1 build)
\item Java (latest JDK 1.7 from Oracle)
\item \LaTeX{} (Mik\TeX{})
\item Python (latest 2.7 from python.org; 64-bit recommended)
\item \pkg{genshi}
\item \pkg{sphinx}
\item Visual Studio (2010, 2012 or 2013; Full or Express edition)
\end{itemize}
\end{frame}

\begin{frame}
  \frametitle{Windows installation (libraries)}

For python, either download separate installers for each packages, or
install \pkg{setuptools} and \pkg{pip} for Python, then \cmd{pip
  install} needed packages; ensure any downloaded packages are 64-bit
if using 64-bit python)
\bigskip

Download and build \pkg{gtest} using \cmd{cmake} (no installation
required)
\bigskip

Build and install the following by hand (for Bio-Formats):
  \begin{columns}
    \begin{column}{.5\linewidth}
      \begin{itemize}
      \item[] \pkg{boost}
      \item[] \pkg{hdf5}
      \item[] \pkg{icu}
      \end{itemize}
    \end{column}
    \begin{column}{.5\linewidth}
      \begin{itemize}
      \item[] \pkg{tiff}
      \item[] \pkg{xerces}
      \item[] \pkg{zlib}
      \end{itemize}
    \end{column}
  \end{columns}
…and possibly more—we haven't yet done a Bio-Formats C++ build on Windows.
\end{frame}

\subsection{Obtaining packages}

\begin{frame}
  \frametitle{Obtaining packages by hand}
  \scriptsize
  \begin{itemize}
  \item Google Test \dotfill \href{https://code.google.com/p/googletest/}{(website)} \href{https://code.google.com/p/googletest/downloads/detail?name=gtest-1.7.0.zip}{(download zip)} \href{http://googletest.googlecode.com/svn/tags/release-1.7.0}{(svn tag)}
  \item CMake \dotfill \href{http://cmake.org/}{(website)} \href{http://cmake.org/cmake/resources/software.html}{(download)}
  \item Java \dotfill \href{http://www.oracle.com/technetwork/java/javase/downloads/jdk7-downloads-1880260.html}{(JDK7 download)}
  \item Visual Studio \dotfill \href{https://my.dundee.ac.uk/webapps/blackboard/content/listContent.jsp?course\_id=\_28436\_1\&content_id=\_3952234\_1}{(Dundee staff)} \href{http://www.visualstudio.com/downloads/download-visual-studio-vs\#d-express-windows-desktop}{(Express download)}
  \item Ant \dotfill \href{http://ant.apache.org/}{(website)} \href{http://ant.apache.org/bindownload.cgi}{(download)}
  \item Git \dotfill \href{http://www.git-scm.com/}{(website)} \href{http://www.git-scm.com/downloads}{(download)}
  \item Ice \dotfill \href{http://zeroc.com/}{(website)} \href{http://zeroc.com/download.html\#win32\_msi}{(download)}
  \item Python \dotfill \href{https://www.python.org/}{(website)} \href{https://www.python.org/download/releases/2.7.8/}{(download)} \href{http://www.lfd.uci.edu/\~gohlke/pythonlibs/}{(extra packages)}
  \item \LaTeX{} \dotfill \href{https://www.tug.org/texlive/}{(\TeX{}Live)} \href{https://www.tug.org/texlive/quickinstall.html}{(\TeX{}Live install)} \href{http://www.miktex.org/}{(Mik\TeX{} website)} \href{http://www.miktex.org/download}{(Mik\TeX{} download)}
  \item Doxygen \dotfill \href{http://www.stack.nl/\~dimitri/doxygen/}{(website)} \href{http://www.stack.nl/\~dimitri/doxygen/download.html}{(download)}
  \item Graphviz \dotfill \href{http://graphviz.org/}{(website)} \href{http://graphviz.org/Download\_windows.php}{(download)}
  \end{itemize}
\end{frame}

\subsection{Configuration}

\begin{frame}
  \frametitle{System configuration}
  \begin{itemize}
    \item In general, none of the tools should require any configuration
      \item \LaTeX{} may require local font configuration to make the \TeX{} Gyre fonts available.
        \begin{itemize}
          \item Linux and FreeBSD: Use the provided \pkg{fontconfig}
            template or create your own
          \item MacOS X: Add to system using FontBook
          \item Windows: May need adding to the system fonts if not
            found automatically
        \end{itemize}
  \end{itemize}
\end{frame}

\begin{frame}
  \frametitle{Environment configuration}
  \begin{itemize}
    \item Primarily needed on Windows
    \item Rather than setting globally, make a batch file which can set up the environment.
    \item Activate a python virtualenv if needed
    \item Ensure that all tools are on the user \texttt{PATH}
      \begin{itemize}
        \item \cmd{ant}, \cmd{cmake}, \cmd{doxygen}, \cmd{dot}, \cmd{git}, \cmd{python}, \cmd{java}, \cmd{sphinx}, \cmd{xelatex}
      \end{itemize}
    \item Set \texttt{CMAKE\_PREFIX\_PATH} if some libraries and tools are not on the default search path.
    \item Not all tools need to be on the default path; some will be discovered automatically by \cmd{cmake}
    \item No need to use a special Visual Studio shell when using \cmd{cmake}
  \end{itemize}
\end{frame}

\section{Building C++ code}

\subsection{Build systems}

\begin{frame}
  \frametitle{Available build systems}
  There are many available build systems, which include:

  \begin{itemize}
  \item Make and GNU Make
  \item GNU Autotools
  \item CMake
  \item Qt \cmd{qmake}
  \item SCons
  \item Jam / BJam
  \item Ant / Maven / Gradle
  \end{itemize}
\end{frame}

\subsection{cmake introduction}

\begin{frame}
  \frametitle{\cmd{cmake} overview}
  \medskip
  \centering
  \includegraphics[width=0.65\textwidth]{cmake-flow}
\end{frame}

\begin{frame}
  \frametitle{\cmd{cmake} features}

  \begin{itemize}
  \item \cmd{cmake} is a generic cross-platform build system
  \item \cmd{cmake} generates build files for a large number of common
    build systems
  \item On FreeBSD, Linux and MacOS X, \cmd{make} \texttt{Makefile}s will be used
  \item On Windows with Visual Studio, \cmd{msbuild} \texttt{.sln}
    solution files will be used
  \item Eclipse, Sublime Text, Kate, Code::Blocks or several other
    IDEs or build systems may be used instead, if desired
  \end{itemize}
\end{frame}

\subsection{cmake demonstration}

\begin{frame}
  \frametitle{Using \cmd{cmake} (live demo)}
  \begin{block}{Basic cmake usage}
    \begin{itemize}
      \item Basic options
      \item Available generators
    \end{itemize}
  \end{block}
  \begin{block}{Building \pkg{gtest} on MacOS X}
    \begin{itemize}
      \item Running cmake
      \item Building
    \end{itemize}
  \end{block}
\end{frame}

\begin{frame}
  \frametitle{Using \cmd{cmake} (live demo)}
  \begin{block}{Building Bio-Formats on MacOS X}
    \begin{itemize}
      \item Running cmake
      \item Cache variables
      \item Building
      \item Testing
      \item Installing
    \end{itemize}
  \end{block}
  \begin{block}{Building OmeroCpp on Windows}
    \begin{itemize}
      \item Running cmake
      \item Building
      \item Installing
    \end{itemize}
  \end{block}
\end{frame}

\section{Examples}
\subsection{gtest on Unix}

\begin{frame}[fragile]
  \frametitle{Building \pkg{gtest} on U\textsc{nix}}
  Build from downloaded zip:

  \begin{semiverbatim}
% cd /tmp
% unzip ~/Downloads/gtest-1.7.0.zip
% cd gtest-1.7.0
% cmake .
% make
\end{semiverbatim}

This is used with other builds by setting the \env{GTEST\_ROOT}
environment variable or the \opt{GTEST\_ROOT} \cmd{cmake} cache
variable.

\end{frame}

\begin{frame}[fragile]
  \frametitle{Building \pkg{gtest} on Debian or Ubuntu}
  Build using installed sources and headers from the \pkg{libgtest-dev} package:

  \begin{semiverbatim}
% cd /tmp
% mkdir gtest
% cd gtest
% cmake /usr/src/gtest
% make
\end{semiverbatim}

This is used with other builds by setting the \env{GTEST\_ROOT}
environment variable or the \opt{GTEST\_ROOT} \cmd{cmake} cache
variable.
\end{frame}

\subsection{Bio-Formats on Unix}

\begin{frame}[fragile]
  \frametitle{Building Bio-Formats on U\textsc{nix} (1)}
  \scriptsize
  Building from git or release zip:

  Configure the build:

  \begin{semiverbatim}
% mkdir /tmp/bfbuild
% cd /tmp/bfbuild
% cmake -DGTEST_ROOT=/tmp/gtest /path/to/bioformats
\end{semiverbatim}

  Show cache variables and advanced cache variables which may be used to customise the build:

  \begin{semiverbatim}
% cmake -LH
% cmake -LAH
\end{semiverbatim}

Run the build with either of:

  \begin{semiverbatim}
% make [VERBOSE=1]
% cmake --build .
\end{semiverbatim}

Build the API reference documentation with either of:

  \begin{semiverbatim}
% make doc
% cmake --build . --target doc
\end{semiverbatim}
\end{frame}

\begin{frame}[fragile]
  \frametitle{Building Bio-Formats on U\textsc{nix} (2)}
  \scriptsize

Run the unit tests with any of:

  \begin{semiverbatim}
% make test
% cmake --build . --target test
% ctest [-V]
\end{semiverbatim}

Individual tests may be run by hand:

  \begin{semiverbatim}
% cpp/test/ome-bioformats/pixelbuffer
% cpp/test/ome-bioformats/pixelbuffer --gtest_help
\end{semiverbatim}

Use \opt{--gtest\_help} to list test options.  Useful when debugging
to run specific tests or subsets of the tests.
\end{frame}

\begin{frame}[fragile]
  \frametitle{Building Bio-Formats on U\textsc{nix} (3)}
  \scriptsize

Install the build with either of:

  \begin{semiverbatim}
% make install [VERBOSE=1] [DESTDIR=/staging/path]
% cmake --build . --target install
\end{semiverbatim}

By default, this will install into \opt{CMAKE\_INSTALL\_PREFIX} which
will default to \path{/usr/local}.  Use \opt{DESTDIR} to install into
an alternative prefixed location, which is useful for testing and
packaging for release.
\end{frame}

\subsection{OmeroCpp on Unix}

\begin{frame}[fragile]
  \frametitle{Building OmeroCpp on U\textsc{nix} (1)}
  \scriptsize
  Building from git or release zip:

  Configure the build. optionally showing Ice autodetection diagnostics:

  \begin{semiverbatim}
% mkdir /tmp/ocppbuild
% cd /tmp/ocppbuild
% cmake -DGTEST_ROOT=/tmp/gtest [-DIce_HOME=/path/to/ice] \\
  [-DIce_DEBUG=ON] /path/to/openmicroscopy
\end{semiverbatim}

  Show cache variables and advanced cache variables which may be used to customise the build:

  \begin{semiverbatim}
% cmake -LH
% cmake -LAH
\end{semiverbatim}

Run the build with either of:

  \begin{semiverbatim}
% make [VERBOSE=1]
% cmake --build .
\end{semiverbatim}
\end{frame}

\begin{frame}[fragile]
  \frametitle{Building OmeroCpp on U\textsc{nix} (2)}
  \scriptsize
Alternatively, it is possible to build in the openmicroscopy tree
directly:

  \begin{semiverbatim}
% ./build.py
% ./build.py build-cpp -Dcmake.opts="cmake options"
\end{semiverbatim}

However, passing in cmake options and using different generators is
much more difficult and more fragile with this method.
\end{frame}


\begin{frame}[fragile]
  \frametitle{Building OmeroCpp on U\textsc{nix} (3)}
  \scriptsize

Run the unit tests with any of:

  \begin{semiverbatim}
% make test
% cmake --build . --target test
% ctest [-V]
\end{semiverbatim}

Note that \env{ICE\_CONFIG} needs setting with the details of a
running OMERO server which the unit and integration tests can connect
to for testing against.
\bigskip

Individual tests may be run by hand:

  \begin{semiverbatim}
% components/tools/OmeroCpp/test/unit/unit
% components/tools/OmeroCpp/test/unit/unit --gtest_help
\end{semiverbatim}

Use \opt{--gtest\_help} to list test options.  Useful when debugging
to run specific tests or subsets of the tests.
\end{frame}

\begin{frame}[fragile]
  \frametitle{Building OmeroCpp on U\textsc{nix} (4)}
  \scriptsize

Install the build with either of:

  \begin{semiverbatim}
% make install [VERBOSE=1] [DESTDIR=/staging/path]
% cmake --build . --target install
\end{semiverbatim}

By default, this will install into \opt{CMAKE\_INSTALL\_PREFIX} which
will default to \path{/usr/local}.  Use \opt{DESTDIR} to install into
an alternative prefixed location, which is useful for testing and
packaging for release.
\end{frame}

\subsection{gtest and OmeroCpp on Windows}

\begin{frame}[fragile]
  \frametitle{Windows environment}
  \scriptsize
  I set up the environment with a custom batch file:

  \begin{semiverbatim}
set "ICE_HOME=C:\\Program Files (x86)\\ZeroC\\Ice-3.5.1"
set "PATH=\%ICE_HOME\%\\bin\\vc110\\x64;C:\\Program Files (x86)\\CMake\\bin;%PATH%"
c:\\venv\\27\\scripts\\activate
\end{semiverbatim}

I also have Ant, Git, Java (JDK), and LaTeX on the default \env{PATH}.
However, these could also be included in the custom batch file.

I use
\href{https://github.com/cbucher/console/wiki/Downloads}{ConsoleZ}
with custom tabs which source different batch files to create
different environments.  For the above, I use the following command to
set up a custom OMERO tab:

  \begin{semiverbatim}
C:\\Windows\\System32\\cmd /k C:\\Users\\rleigh\\bin\\omeroenv.bat
\end{semiverbatim}

Note that the Ice setup is only required for running \cmd{build.py};
it is optional for direct use of \cmd{cmake}.
\end{frame}

\begin{frame}[fragile]
  \frametitle{Building \pkg{gtest} on Windows}
  \scriptsize
  Download and unpack \pkg{gtest}, then run:

  \begin{semiverbatim}
> set CL=/D_VARIADIC_MAX=10
> cd c:\\Users\\rleigh\\gtest-1.7.0
> cmake -G "Visual Studio 11 2012 Win64" .
> cmake --build .
\end{semiverbatim}

The \texttt{\_VARIADIC\_MAX=10} define works around a lack of
variadic templates in this version of Visual Studio; may affect other
Visual Studio versions.  Leave set for the remaining steps.
\end{frame}

\begin{frame}[fragile]
  \frametitle{Building OmeroCpp on Windows}
  \scriptsize

  Note: OmeroCpp building on Windows is a work in progress and not get
  completely finished.

  \bigskip
  Note: starting from a \emph{clean} and up-to-date \emph{develop}
  branch of \pkg{openmicroscopy.git} located in
  \path{c:\Users\rleigh\openmicroscopy}.

  \begin{semiverbatim}
> mkdir c:\\Users\\rleigh\\ocppbuild
> cd c:\\Users\\rleigh\\ocppbuild
> cmake -G "Visual Studio 11 2012 Win64" \\
  -DGTEST_ROOT=C:\\Users\\rleigh\\gtest-1.7.0 \\
  -DGTEST_LIBRARY=C:\\Users\\rleigh\\gtest-1.7.0\\Debug\\gtest.lib \\
  -DGTEST_MAIN_LIBRARY=C:\\Users\\rleigh\\gtest-1.7.0\\Debug\\gtest_main.lib \\
  ..\\openmicroscopy
> cmake --build .
\end{semiverbatim}

After running \cmd{cmake}, it's also possible to open the solution
file in Visual Studio and build from inside the application.
\end{frame}

\appendix

\section[]{Acknowledgements}

\frame{
  \frametitle{Acknowledgements}
  \parbox[t]{0.45\textwidth}{
    \begin{itemize}
    \item OME Team, Dundee
      \begin{itemize}
      \item Jason Swedlow
      \item Jean-Marie Burel
      \item Mark Carroll
      \item Andrew Patterson
      \item …and the rest of the team
      \end{itemize}
    \end{itemize}
  }
  \parbox[t]{0.45\textwidth}{
    \begin{itemize}
    \item Micron, Oxford
      \begin{itemize}
      \item Douglas Russell
      \end{itemize}
    \item Glencoe Software
      \begin{itemize}
      \item Melissa Linkert
      \item Josh Moore
      \end{itemize}
    \end{itemize}
  }

  \begin{center}
    \vcenteredhbox{\includegraphics[width=0.25\textwidth]{ome}} \hfill
    \vcenteredhbox{\includegraphics[width=0.2\textwidth]{dundee}}\hfill
    \vcenteredhbox{\includegraphics[width=0.5\textwidth]{wellcome}}
  \end{center}
}

\end{document}
