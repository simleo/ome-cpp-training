\documentclass{beamer}

\newcommand{\cmd}[1]{\textbf{\texttt{#1}}}
\newcommand{\pkg}[1]{\texttt{#1}}
\newcommand{\env}[1]{\texttt{#1}}
\newcommand{\opt}[1]{\textsl{#1}}

\usepackage{beamerthemesplit}
\beamertemplatenavigationsymbolsempty
\setbeamertemplate{footline}[frame number]{}

\usepackage{graphicx}
\usepackage{color}
\usepackage{listings}
\usepackage{verbatim}
\usepackage{setspace}
\usepackage{url}
\usepackage{hyperref}

\usepackage{fontspec}
\setmonofont{Latin Modern Mono}
\setsansfont{TeX Gyre Heros}

\lstset{breakatwhitespace=true,
language=C++,
basicstyle=\footnotesize\ttfamily,
keywordstyle=\color{blue}\ttfamily,
stringstyle=\color{red}\ttfamily,
commentstyle=\color{brown}\ttfamily,
morecomment=[l][\color{magenta}]{\#}keepspaces=true,
breaklines=true,
tabsize=3,
showstringspaces=false,
extendedchars=true,
frame=single,
escapeinside=§§
}

\newcommand*{\vcenteredhbox}[1]{\begingroup
\setbox0=\hbox{#1}\parbox{\wd0}{\box0}\endgroup}

\usepackage{array}
\usepackage{tikz}
\usetikzlibrary{arrows.meta,calc,decorations.markings,matrix,bending}

\newcommand*{\abox}[1]{\framebox{\hbox to 0.5cm{\texttt\{{\\hss#1\hss}}}}

\newcommand{\filename}[1]{\texttt{#1}}
\newcommand{\program}[1]{\texttt{#1}}
\newcommand{\variable}[1]{\textsl{#1}}
\newcommand{\class}[1]{\texttt{#1}}
\newcommand{\function}[1]{\texttt{#1()}}
\newcommand{\type}[1]{\texttt{#1}}
\newcommand{\code}[1]{\texttt{#1}}

\lstdefinelanguage{XML}
{
  basicstyle=\ttfamily\footnotesize,
  morestring=[b]",
  moredelim=[s][\bfseries\color{black!30!red}]{<}{\ },
  moredelim=[s][\bfseries\color{black!30!red}]{</}{>},
  moredelim=[l][\bfseries\color{black!30!red}]{/>},
  moredelim=[l][\bfseries\color{black!30!red}]{>},
  morecomment=[s]{<?}{?>},
  morecomment=[s]{<!--}{-->},
  commentstyle=\color{black!30!green},
  stringstyle=\color{blue},
  identifierstyle=\color{red}
}

\newenvironment{conditions}
  {\par\vspace{\abovedisplayskip}\noindent\begin{tabular}{>{$}l<{$} @{${}={}$} l}}
  {\end{tabular}\par\vspace{\belowdisplayskip}}

\title{Multi-dimensional Arrays}
\author{Roger Leigh}
\date{April 2016}

\begin{document}

\begin{frame}[plain,fragile]
  \titlepage
    \begin{tikzpicture}
      \node[minimum width=\paperwidth, minimum height=\paperheight, anchor=north west] (a) {
      \begin{tikzpicture}[x=1mm,y=-1mm]
        \pgfmathsetmacro\ax{cos(33)*0.8}
        \pgfmathsetmacro\bx{cos(67)*0.8}

      \foreach \col in {0, 1, 2, 3, 4, 5} {
        \foreach \row in {0, 1, 2, 3, 4, 5} {
          \pgfmathsetmacro\acx{\row-20+(((\ax*10)+11+\row)*\col)}
          \pgfmathsetmacro\acy{30-((\bx*10)+11+\col)*\row}

          \foreach \x in {0,1,2,3,4,5,6,7,8,9,10} {
            \draw [thin, red] (10+\acx+\x, 10+\acy) -- (10+\acx+\x,20+\acy);   
            \draw [thin, green] (10+\acx, 10+\acy+\x) -- (20+\acx,10+\acy+\x);   

            \draw [thin, green] (20+\acx,10+\acy+\x) -- ({20+\acx+(\ax * 10)},{10+\acy+\x-(\bx * 10)});   
            \draw [thin, blue] ({20+\acx+(\ax*\x)},{10+\acy-(\bx*\x)}) -- ({20+\acx+(\ax*\x)},{10+\acy+10-(\bx * \x)});   

            \draw [thin, red] (10+\acx+\x,10+\acy) -- ({10+\acx+\x+(\ax * 10)}, {10+\acy-(\bx * 10)});   
            \draw [thin, blue] ({10+\acx+(\ax*\x)}, {10+\acy-(\bx*\x)}) -- ({10+\acx+10+(\ax * \x)}, {10+\acy-(\bx*\x)});   
          }
        }
      };
    \end{tikzpicture}
      };
    \end{tikzpicture}
\end{frame}

\begin{frame}
  \frametitle{History}
  \begin{itemize}
  \item \href{https://trac.openmicroscopy.org/ome/ticket/3523}{Trac ticket 3523}
  \item \href{http://www.openmicroscopy.org/site/community/minutes/meetings/june-2010-paris-users-meeting/meeting-notes}{2010 Paris User's meeting}
  \item Also considered in earlier meetings back to at least 2008
  \end{itemize}
\end{frame}

\begin{frame}
  \frametitle{Current model}
  \begin{itemize}
  \item Data model \texttt{DimensionOrder}: X Y Z T C
    \begin{itemize}
      \item \texttt{XYZTC}
      \item \texttt{XYZCT}
      \item \texttt{XYTZC}
      \item \texttt{XYTCX}
      \item \texttt{XYCZT}
      \item \texttt{XYCTZ}
    \end{itemize}
    \pause
  \item Also subchannel/samples
  \item Extra ``dimensions'' hacked in with Modulo
  \item XY always ordered first
  \item Enumeration can't scale
  \item Fundamentally limited
  \end{itemize}
\end{frame}

\begin{frame}[fragile]
  \frametitle{Current API}
  \begin{block}{Plane index calculations}
    Compute index from Z, C and T coordinates:
  \begin{lstlisting}[language=Java]
int
getIndex(String order,
         int zSize, int cSize, int tSize, int num,
         int z, int c, int t)
  \end{lstlisting}
  Compute Z, C and T coordinates from an index:
  \begin{lstlisting}[language=Java]
int[]
getZCTCoords(String order,
             int zSize, int cSize, int tSize, int num,
             int index);
  \end{lstlisting}
  \begin{itemize}
  \item Limited to three fixed dimensions
  \item Implementation is inefficient (string parsing and validity checks)
  \end{itemize}
  \end{block}
\end{frame}

\begin{frame}
  \frametitle{Describing a dimension}
  \begin{columns}[c]
    \column{.5\textwidth}
    \begin{block}{Core metadata}
      \begin{itemize}
      \item extent (size)
        \pause
      \item order (logical)
        \pause
      \item order (storage)
      \item direction (storage)
      \end{itemize}
    \end{block}
    \column{.5\textwidth}
    \pause
    \begin{block}{Extended metadata}
      \begin{itemize}
      \item dimension type
      \item unit type e.g. physical size unit
      \item unit value range e.g. physical size
      \item labels
      \item ... essentially some of the Modulo metadata
      \end{itemize}
    \end{block}
  \end{columns}
\end{frame}

\begin{frame}[fragile]
  \frametitle{Current API}
  \begin{lstlisting}[language=Java]
byte[] openBytes(int no,
                 byte[] buf,
                 int x, int y, int w, int h);
  \end{lstlisting}
  \begin{block}{Image index}
    \begin{itemize}
    \item Plane number for desired ZTC coordinate
    \item Requires calculation by caller
    \item Exposes internal implementation details of file format and/or reader
    \item May not be real if file format is not planar
    \end{itemize}
  \end{block}
  \begin{tikzpicture}[remember picture,overlay]
    \node at (current page.north west) {
      \begin{tikzpicture}[y=-1cm,remember picture,overlay]
        \draw [very thick, black!30!red] (4.8,2.3) ellipse (0.75 cm and 0.25 cm);
        \draw [-{Latex[scale=1.5]}, very thick, black!30!red] (2,3.8) -- (4.2,2.4);
      \end{tikzpicture}
    };
  \end{tikzpicture}
\end{frame}

\begin{frame}[fragile]
  \frametitle{Current API}
  \begin{lstlisting}[language=Java]
byte[] openBytes(int no,
                 byte[] buf,
                 int x, int y, int w, int h);
  \end{lstlisting}
  \begin{block}{Pixel buffer}
    \begin{itemize}
    \item Raw byte array
    \item No pixel type
    \item No dimension sizes and order
    \item No padding information
    \item Difficult to process or delegate to other code
    \end{itemize}
  \end{block}
  \begin{tikzpicture}[remember picture,overlay]
    \node at (current page.north west) {
      \begin{tikzpicture}[y=-1cm,remember picture,overlay]
        \draw [very thick, black!30!red] (5.15,2.65) ellipse (1.1 cm and 0.3 cm);
        \draw [-{Latex[scale=1.5]}, very thick, black!30!red] (2,3.8) -- (4.15,2.8);
      \end{tikzpicture}
    };
  \end{tikzpicture}
\end{frame}

\begin{frame}[fragile]
  \frametitle{Current API}
  \begin{lstlisting}[language=Java]
byte[] openBytes(int no,
                 byte[] buf,
                 int x, int y, int w, int h);
  \end{lstlisting}
  \begin{block}{Starting coordinates and extents}
    \begin{itemize}
    \item Separate parameter for every dimension coordinate
    \item Separate parameter for every dimension extent
    \item No means of using a range rather than an extent
    \item Limited to specifying regions on a 2D plane
    \item No generality
    \end{itemize}
  \end{block}
  \begin{tikzpicture}[remember picture,overlay]
    \node at (current page.north west) {
      \begin{tikzpicture}[y=-1cm,remember picture,overlay]
        \draw [very thick, black!30!red] (5.4,3.02) ellipse (1.3 cm and 0.25 cm);
        \draw [very thick, black!30!red] (8.05,3.02) ellipse (1.3 cm and 0.25 cm);
        \draw [-{Latex[scale=1.5]}, very thick, black!30!red] (3.5,3.8) -- (4.25,3.1);
        \draw [-{Latex[scale=1.5]}, very thick, black!30!red] (6.25,3.8) -- (6.95,3.1);
      \end{tikzpicture}
    };
  \end{tikzpicture}
\end{frame}

\begin{frame}[fragile]
  \frametitle{Addressing: One dimension}

  \begin{center}

  \begin{tikzpicture}[y=-1cm]
    \foreach \x in {0,1,2,3,4,5}
      \draw[gray,thick] (\x,0) rectangle (\x+1,1);
    \draw[black,very thick] (0,0) rectangle (6,1);
    \foreach \x in {0,1,2,3,4,5}
      \draw (\x+0.5, 0.5) node[black!50!blue] {\x};
    \draw (0, 0.5) node[anchor=east,black!50!blue] {Index};

    \onslide<2->{
    \foreach \x in {0,1,2,3,4,5}
      \draw[gray,thick] (\x,1) rectangle (\x+1,2);
    \draw[black,very thick] (0,1) rectangle (6,2);
    \foreach \x in {0,1,2,3,4,5}
      \draw (\x+0.5, 1.5) node[black!50!green] {\x};
    \draw (0, 1.5) node[anchor=east,black!50!green] {Offset};
    }
    \onslide<3->{
    \draw [-{Latex[scale=1.5]}, thick, black!30!red] (0.5,2.6) -- (0.5,2);   
    \draw (0.5,2.5) node[anchor=north, black!30!red] {base=0};

    \draw [-{Latex[scale=1.5,flex=1]}, thick, black!30!red] (3.5,2) arc[radius=0.5, start angle=-180, end angle=-360];
    \draw (4,2.5) node[anchor=north, black!30!red] {stride=1};
    }
  \end{tikzpicture}

  \onslide<4->{
  \begin{equation*}
    \mathrm{Offset} = \mathrm{base} + (\mathrm{Index} \cdot \mathrm{stride})
  \end{equation*}
  \begin{equation*}
    \mathrm{Offset} = \mathrm{Index}
  \end{equation*}
  }
  \end{center}
\end{frame}

\begin{frame}[fragile]
  \frametitle{Addressing: One dimension (reversed)}
  \begin{center}
  \begin{tikzpicture}[y=-1cm]
    \foreach \x in {0,1,2,3,4,5}
      \draw[gray,thick] (\x,0) rectangle (\x+1,1);
    \draw[black,very thick] (0,0) rectangle (6,1);
    \foreach \x in {0,1,2,3,4,5}
      \draw (\x+0.5, 0.5) node[black!50!blue] {\x};
    \draw (0, 0.5) node[anchor=east,black!50!blue] {Index};

    \foreach \x in {0,1,2,3,4,5}
      \draw[gray,thick] (\x,1) rectangle (\x+1,2);
    \draw[black,very thick] (0,1) rectangle (6,2);
    \foreach \x in {0,1,2,3,4,5}
      \draw[black!50!green] node at (\x+0.5, 1.5) {%
    \pgfmathparse{5-\x}%
    \pgfmathprintnumber[assume math mode=true]{\pgfmathresult}%
    };
    \draw (0, 1.5) node[anchor=east,black!50!green] {Offset};

    \draw [-{Latex[scale=1.5]}, thick, black!30!red] (5.5,2.6) -- (5.5,2);   
    \draw (5.5,2.5) node[anchor=north, black!30!red] {base=5};

    \draw [-{Latex[scale=1.5,flex=1]}, thick, black!30!red] (2.5,2) arc[radius=0.5, start angle=-360, end angle=-180];
    \draw (2,2.5) node[anchor=north, black!30!red] {stride=-1};
  \end{tikzpicture}

  \onslide<2->{
  \begin{equation*}
    \mathrm{Offset} = \mathrm{base} + (\mathrm{Index} \cdot \mathrm{stride})
  \end{equation*}
  \begin{equation*}
    \mathrm{Offset} = 5 + (\mathrm{Index} \cdot -1)
  \end{equation*}
  \begin{equation*}
    \mathrm{Offset} = 5 - \mathrm{Index}
  \end{equation*}
  }
  \end{center}
\end{frame}

\begin{frame}[fragile]
  \frametitle{Addressing: Two dimensions (row-major)}
  \begin{center}
  \begin{tikzpicture}[y=-1cm]
    \foreach \x in {0,1,2,3,4}
      \draw[gray,thick] (\x,-1) rectangle (\x+1,0);
    \foreach \y in {0,1,2}
      \draw[gray,thick] (-1,\y) rectangle (0,\y);
    \draw[black,very thick] (0,-1) rectangle (5,0);
    \draw[black,very thick] (-1,0) rectangle (0,3);
    \draw[black,very thick] (0,0) rectangle (5,3);
    \foreach \x in {0,1,2,3,4}
      \draw (\x+0.5, -0.5) node[black!50!blue] {\x};
    \foreach \y in {0,1,2}
      \draw (-0.5,\y+0.5) node[black!50!blue] {\y};
    \draw (2.5, -1) node[anchor=south,black!50!blue] {x Index};
    \draw (-1, 1.5) node[anchor=east,black!50!blue] {\rotatebox{90}{y Index}};

    \onslide<2->{
    \foreach \x in {0,1,2,3,4}
      \foreach \y in {0,1,2} {
        \draw[gray,thick] (\x,\y) rectangle (\x+1,\y+1);
        \draw[black!50!green] node at (\x+0.5, \y+0.5) {%
          \pgfmathparse{\x+(5*\y)}%
          \pgfmathprintnumber[assume math mode=true]{\pgfmathresult}%
        };
      }
      \draw[black,very thick] (0,0) rectangle (5,3);
      \draw (5, 2.5) node[anchor=west,black!50!green] {Offset};
    }
    \onslide<3->{
    \draw [-{Latex[scale=1.5]}, thick, black!30!red] (-0.5,-0.5) -- (0,0);   
    \draw (-0.5,-0.5) node[anchor=south east, black!30!red] {base=0};

    \draw [-{Latex[scale=1.5,flex=1]}, thick, black!30!red] (0.5,3) arc[radius=0.5, start angle=-180, end angle=-360];
    \draw (1,3.5) node[anchor=north, black!30!red] {x stride=1};

    \draw [-{Latex[scale=1.5,flex=1]}, thick, black!30!red] (5,0.5) arc[radius=0.5, start angle=-90, end angle=90];
    \draw (5.5,1) node[anchor=west, black!30!red] {y stride=5};
    }
  \end{tikzpicture}

  \onslide<4->{
  \begin{equation*}
    \mathrm{Offset} = \mathrm{base} + (\mathrm{Index}_x \cdot \mathrm{stride}_x) + (\mathrm{Index}_y \cdot \mathrm{stride}_y)
  \end{equation*}
  \begin{equation*}
    \mathrm{Offset} = \mathrm{Index}_x + (\mathrm{Index}_y \cdot 5)
  \end{equation*}
  }
  \end{center}
\end{frame}

\begin{frame}[fragile]
  \frametitle{Addressing: Two dimensions (column-major)}
  \begin{center}
  \begin{tikzpicture}[y=-1cm]
    \foreach \x in {0,1,2,3,4}
      \draw[gray,thick] (\x,-1) rectangle (\x+1,0);
    \foreach \y in {0,1,2}
      \draw[gray,thick] (-1,\y) rectangle (0,\y);
    \draw[black,very thick] (0,-1) rectangle (5,0);
    \draw[black,very thick] (-1,0) rectangle (0,3);
    \draw[black,very thick] (0,0) rectangle (5,3);
    \foreach \x in {0,1,2,3,4}
      \draw (\x+0.5, -0.5) node[black!50!blue] {\x};
    \foreach \y in {0,1,2}
      \draw (-0.5,\y+0.5) node[black!50!blue] {\y};
    \draw (2.5, -1) node[anchor=south,black!50!blue] {x Index};
    \draw (-1, 1.5) node[anchor=east,black!50!blue] {\rotatebox{90}{y Index}};

    \foreach \x in {0,1,2,3,4}
      \foreach \y in {0,1,2} {
        \draw[gray,thick] (\x,\y) rectangle (\x+1,\y+1);
        \draw[black!50!green] node at (\x+0.5, \y+0.5) {%
          \pgfmathparse{(\x*3)+\y}%
          \pgfmathprintnumber[assume math mode=true]{\pgfmathresult}%
        };
      }
      \draw[black,very thick] (0,0) rectangle (5,3);
      \draw (5, 2.5) node[anchor=west,black!50!green] {Offset};

    \onslide<2->{
    \draw [-{Latex[scale=1.5]}, thick, black!30!red] (-0.5,-0.5) -- (0,0);   
    \draw (-0.5,-0.5) node[anchor=south east, black!30!red] {base=0};

    \draw [-{Latex[scale=1.5,flex=1]}, thick, black!30!red] (0.5,3) arc[radius=0.5, start angle=-180, end angle=-360];
    \draw (1,3.5) node[anchor=north, black!30!red] {x stride=3};

    \draw [-{Latex[scale=1.5,flex=1]}, thick, black!30!red] (5,0.5) arc[radius=0.5, start angle=-90, end angle=90];
    \draw (5.5,1) node[anchor=west, black!30!red] {y stride=1};
    }
  \end{tikzpicture}

  \onslide<3->{
  \begin{equation*}
    \mathrm{Offset} = \mathrm{base} + (\mathrm{Index}_x \cdot \mathrm{stride}_x) + (\mathrm{Index}_y \cdot \mathrm{stride}_y)
  \end{equation*}
  \begin{equation*}
    \mathrm{Offset} = (\mathrm{Index}_x \cdot 3) + \mathrm{Index}_y
  \end{equation*}
  }
  \end{center}
\end{frame}

\begin{frame}[fragile]
  \frametitle{Addressing: Two dimensions (column-major, yx order)}
  \begin{center}
  \begin{tikzpicture}[y=-1cm]
    \foreach \x in {0,1,2,3,4}
      \draw[gray,thick] (\x,-1) rectangle (\x+1,0);
    \foreach \y in {0,1,2}
      \draw[gray,thick] (-1,\y) rectangle (0,\y);
    \draw[black,very thick] (0,-1) rectangle (5,0);
    \draw[black,very thick] (-1,0) rectangle (0,3);
    \draw[black,very thick] (0,0) rectangle (5,3);
    \foreach \x in {0,1,2,3,4}
      \draw (\x+0.5, -0.5) node[black!50!blue] {\x};
    \foreach \y in {0,1,2}
      \draw (-0.5,\y+0.5) node[black!50!blue] {\y};
    \draw (2.5, -1) node[anchor=south,black!50!blue] {x Index};
    \draw (-1, 1.5) node[anchor=east,black!50!blue] {\rotatebox{90}{y Index}};

    \foreach \x in {0,1,2,3,4}
      \foreach \y in {0,1,2} {
        \draw[gray,thick] (\x,\y) rectangle (\x+1,\y+1);
        \draw[black!50!green] node at (\x+0.5, \y+0.5) {%
          \pgfmathparse{12+(\x*-3)+\y}%
          \pgfmathprintnumber[assume math mode=true]{\pgfmathresult}%
        };
      }
      \draw[black,very thick] (0,0) rectangle (5,3);
      \draw (5, 2.5) node[anchor=west,black!50!green] {Offset};

    \onslide<2->{
    \draw [-{Latex[scale=1.5]}, thick, black!30!red] (-0.5,-0.5) -- (0,0);   
    \draw (-0.5,-0.5) node[anchor=south east, black!30!red] {base=12};

    \draw [-{Latex[scale=1.5,flex=1]}, thick, black!30!red] (4.5,3) arc[radius=0.5, start angle=-360, end angle=-180];
    \draw (4,3.5) node[anchor=north, black!30!red] {x stride=-3};

    \draw [-{Latex[scale=1.5,flex=1]}, thick, black!30!red] (5,0.5) arc[radius=0.5, start angle=-90, end angle=90];
    \draw (5.5,1) node[anchor=west, black!30!red] {y stride=1};
    }
  \end{tikzpicture}

  \onslide<3->{
  \begin{equation*}
    \mathrm{Offset} = \mathrm{base} + (\mathrm{Index}_x \cdot \mathrm{stride}_x) + (\mathrm{Index}_y \cdot \mathrm{stride}_y)
  \end{equation*}
  \begin{equation*}
    \mathrm{Offset} = 12 + (\mathrm{Index}_x \cdot -3) + \mathrm{Index}_y
  \end{equation*}
  }
  \end{center}
\end{frame}


\begin{frame}
  \frametitle{Logical addressing}
  \begin{columns}[c]
    \column{.5\textwidth}
    \begin{block}{Terms}
    \begin{conditions}
      n & Number of dimensions \\
      C & Coordinate \\
      I & Index \\
      S & Stride \\
    \end{conditions}
    \end{block}
    \column{.5\textwidth}
    \begin{block}{Linear index from logical coordinate}
    \begin{equation*}
      I = \sum\limits_{i=0}^{n-1} (S_i \cdot C_i)
    \end{equation*}
    \end{block}
    \begin{block}{Logical coordinate from linear index}
      For each dimension $i$:
      \begin{equation*}
        C_i = \frac{I}{S_i}
      \end{equation*}

      \begin{equation*}
        I = I \operatorname{mod} S_i
      \end{equation*}
    \end{block}
  \end{columns}
\end{frame}

\begin{frame}
  \frametitle{Storage addressing}
  \begin{columns}[c]
    \column{.5\textwidth}
    \begin{block}{Terms}
    \begin{conditions}
      n & Number of dimensions \\
      E & Dimension extent \\
      C & Coordinate \\
      I & Index \\
      S & Stride \\
      O & Base offset \\
    \end{conditions}
    \end{block}
    \begin{block}{Linear index from logical coordinate}
    \begin{equation*}
      I = O + \sum\limits_{i=0}^{n-1} (S_i \cdot C_i)
    \end{equation*}
    \end{block}
    \column{.5\textwidth}
    \begin{block}{Logical coordinate from linear index}
      For each dimension $i$:
      If ascending:
      \begin{equation*}
        C_i = \frac{I}{S_i}
      \end{equation*}
      \begin{equation*}
        I = I \operatorname{mod} S_i
      \end{equation*}
      If descending:
      \begin{equation*}
        C_i = \frac{(E_i \cdot \lvert S_i \rvert) - S_i - 1}{\lvert S_i \rvert}
      \end{equation*}
      \begin{equation*}
        I = I \operatorname{mod} \lvert S_i \rvert
      \end{equation*}
    \end{block}
  \end{columns}
\end{frame}

\begin{frame}[fragile]
  \frametitle{Addressing: subranges and assignment}
  \begin{center}
  \begin{tikzpicture}[y=-1cm]
    \onslide<1->{
      \draw[black,very thick] (6,0) rectangle (10,5);
    }
    \onslide<2->{
      \draw[gray,thick,fill=white!70!red] (8,2) rectangle (9,3);
    }
    \onslide<3->{
      \draw [thick, black!30!red, dashed] (8,2) -- (6,6);   
      \draw [thick, black!30!red, dashed] (9,3) -- (7,7);   
      \draw[black,very thick,fill=white] (6,6) rectangle (7,7);
    }
    \onslide<4->{
      \draw[black,very thick] (0,0) rectangle (5,4);
      \draw[gray,thick,fill=white!70!red] (1,1) rectangle (2,2);
    }
    \onslide<5->{
      \draw [thick, black!30!red, dashed] (1,2) -- (4,7);   
      \draw [thick, black!30!red, dashed] (2,1) -- (5,6);   
      \draw[black,very thick] (4,6) rectangle (5,7);
    }
    \onslide<6->{
      \draw (5.5,6.5) node {=};
    }
    \onslide<7->{
    \draw [-{Latex[scale=1.5]}, thick, black!30!green] (8.5,2.5) -- (1.5,1.5);   
    }
  \end{tikzpicture}
  \end{center}

\end{frame}

\begin{frame}
  \frametitle{Storage addressing with subranges}
  \begin{columns}
    \column{.5\textwidth}
    \begin{block}{Terms}
    \begin{conditions}
      n & Number of dimensions \\
      E & Dimension extent \\
      C & Coordinate \\
      I & Index \\
      S & Stride \\
      O & Base offset \\
      B & Subrange start offset \\
    \end{conditions}
    \end{block}
    \begin{block}{Linear index from logical coordinate}
    \begin{equation*}
      I = O + \sum\limits_{i=0}^{n-1} (S_i \cdot (C_i + B_i))
    \end{equation*}
    \end{block}
    \column{.5\textwidth}
    \begin{block}{Logical coordinate from linear index}
      For each dimension $i$:\\ \noindent
      If ascending:
      \begin{equation*}
        C_i = \frac{I}{S_i} - B_i
      \end{equation*}
      \begin{equation*}
        I = I \operatorname{mod} S_i
      \end{equation*}
      If descending:
      \begin{equation*}
        C_i = \frac{(E_i \cdot \lvert S_i \rvert) - S_i - 1}{\lvert S_i \rvert} - B_i
      \end{equation*}
      \begin{equation*}
        I = I \operatorname{mod} \lvert S_i \rvert
      \end{equation*}
    \end{block}
  \end{columns}
\end{frame}

\begin{frame}[fragile]
  \frametitle{Proposed model}
  \begin{lstlisting}[language=XML]
<DimensionSpace>
  <Dimension name="S" size=3 position=0 ascending=true/>
  <Dimension name="X" size=1024 position=2 ascending=true/> 
  <Dimension name="Y" size=1024 position=1 ascending=false/>
  ...
</DimensionSpace>
  \end{lstlisting}

  \begin{itemize}
  \item Arbitrary named dimensions
  \item Logical order implicit, storage order explicit
  \item Extra modulo-style attributes
  \item X and Y not special, any order allowed
  \item Modulo no longer needed
  \item Could also add \texttt{chunkSize} attribute to add transparent
    support for tiling/chunking
  \end{itemize}
\end{frame}

\begin{frame}
  \frametitle{Proposed class for calculations}

  \begin{block}{Class \texttt{DimensionSpace}}
    \begin{itemize}
    \item Perform nD calculations
    \item Contains a list of \texttt{Dimension} classes specifying
      dimension extents, ordering and direction
    \item Contains precomputed constants (strides, offsets)
    \item Compute storage offset from logical indices
    \item Compute logical indices from storage offset
    \item The generalised implementation of \texttt{getIndex} and
      \texttt{getCZTCoords}
    \end{itemize}
  \end{block}
\end{frame}

\begin{frame}[fragile]
  \frametitle{Proposed implementation for calculations}
  \begin{block}{Plane index calculations}
    Compute index from Z, C and T coordinates:
  \begin{lstlisting}[language=C++]
index_type
storage_index(const coord_type& coord) const
{
  difference_type pos = _base;

  for (std::vector<Dimension>::size_type d = 0; d < size(); ++d)
    {
      const Dimension& dim(dimension(d));
      const DimensionStorageDetail& detail = _detail.at(d);
      pos += detail.stride * (coord.at(d) + dim.begin);
    }

  return pos;
}
  \end{lstlisting}
  \end{block}
\end{frame}

\begin{frame}[fragile]
  \frametitle{Existing implementation for calculations (Imglib2)}
  \begin{block}{Plane index calculations}
    Compute index from Z, C and T coordinates:
    \begin{lstlisting}[language=Java]
int positionWithOffsetToIndex(final int[] position,
                              final int[] dimensions,
                              final int[] offsets)
{
  final int maxDim = dimensions.length - 1;
  int i = position[ maxDim ] - offsets[ maxDim ];
  for ( int d = maxDim - 1; d >= 0; --d )
    i = i * dimensions[ d ] + position[ d ] - offsets[ d ];
  return i;
}
    \end{lstlisting}
    \begin{itemize}
    \item Avoids stride calculations by iterating backward
    \item Does not cater for reverse progression
    \end{itemize}
  See \filename{util/IntervalIndexer.java}
  \end{block}
\end{frame}

\begin{frame}[fragile]
  \frametitle{Proposed generalised pixel methods}
  \begin{lstlisting}[language=Java]
    PixelBuffer getPixels(PixelBuffer    buf,
                          CoordinateList coords
                          ExtentList     extents);
  \end{lstlisting}
  or
  \begin{lstlisting}[language=Java]
    PixelBuffer getPixels(PixelBuffer    buf,
                          CoordinateList start
                          CoordinateList end);
  \end{lstlisting}
  \begin{block}{Pixel buffer}
    \begin{itemize}
    \item \class{PixelBuffer} provides typed pixel data, with extent sizes
    \item \class{CoordinateList} is the set of indexes into each dimension
    \item \class{ExtentList} is the extent in each dimension
    \item No plane number needed; the coordinate and extent lists
      determine the plane(s) to fetch
    \end{itemize}
  \end{block}
\end{frame}

\begin{frame}
  \frametitle{Implementation order}
  \begin{block}{Key parts}
    \begin{itemize}
    \item Model changes
    \item nD calculations
    \item nd PixelBuffer
    \end{itemize}
  \end{block}
  \begin{block}{Implementation order}
    \begin{itemize}
    \item nD calculations
    \item nd PixelBuffer
    \item Model changes
    \end{itemize}
    The nD calculations and pixel buffer are a superset of the current
    model and can represent the current model perfectly, so the model
    changes can be done last.
  \end{block}
\end{frame}

\appendix

\section[]{Acknowledgements}

\frame{
  \frametitle{Acknowledgements}

  \begin{center}
    \vcenteredhbox{\includegraphics[width=0.25\textwidth]{ome}} \hfill
    \vcenteredhbox{\includegraphics[width=0.2\textwidth]{dundee}}\hfill
    \vcenteredhbox{\includegraphics[width=0.5\textwidth]{wellcome}}
  \end{center}
}

\end{document}
