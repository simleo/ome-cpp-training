\documentclass[handout]{beamer}

\usepackage{beamerthemesplit}
\beamertemplatenavigationsymbolsempty
\setbeamertemplate{footline}[frame number]{}

\usepackage{graphicx}
\usepackage{color}
\usepackage{listings}
\usepackage{verbatim}
\usepackage{setspace}
\usepackage{url}
\usepackage{hyperref}

\usepackage{fontspec}
\setmonofont{Latin Modern Mono}
\setsansfont{TeX Gyre Heros}

\lstset{breakatwhitespace=true,
language=C++,
basicstyle=\footnotesize\ttfamily,
keywordstyle=\color{blue}\ttfamily,
stringstyle=\color{red}\ttfamily,
commentstyle=\color{brown}\ttfamily,
morecomment=[l][\color{magenta}]{\#}keepspaces=true,
breaklines=true,
tabsize=3,
showstringspaces=false,
extendedchars=true,
frame=single,
escapeinside=§§
}

\newcommand*{\vcenteredhbox}[1]{\begingroup
\setbox0=\hbox{#1}\parbox{\wd0}{\box0}\endgroup}


\title{Bio-Formats C++ Conversion}
\subtitle{Differences between C++ and Java}
\author{Roger Leigh}
\date{March 2014\\University of Dundee}

\begin{document}

\begin{frame}[plain]
  \titlepage
  \begin{center}
    \vcenteredhbox{\includegraphics[width=0.25\textwidth]{ome}} \hfill
    \vcenteredhbox{\includegraphics[width=0.2\textwidth]{dundee}}\hfill
    \vcenteredhbox{\includegraphics[width=0.25\textwidth]{wellcome}}
  \end{center}
\end{frame}

\section*{Overview}
\frame{
\frametitle{Overview}
\tableofcontents
}

\section{C++ and Java differences}
\subsection{C++ and Java types}

\begin{frame}[fragile]
  \frametitle{Java types}

  \begin{block}{Primitive types}
    \begin{lstlisting}[language=Java]
int i;     double d;
\end{lstlisting}
    \begin{itemize}
      \pause
    \item No unsigned integer types
    \end{itemize}
  \end{block}
  \pause
  \begin{block}{Classes}
    \begin{lstlisting}[language=Java]
Pixels pixels = new Pixels();
\end{lstlisting}
    \begin{itemize}
      \pause
    \item Classes, all derived from root \texttt{Object}
      \pause
    \item Objects are by reference only
      \pause
    \item Objects and arrays are always allocated with \texttt{new}
      \pause
    \item Destruction is non-deterministic
    \end{itemize}
  \end{block}
\end{frame}

\begin{frame}[fragile]
  \frametitle{Java types}
  \begin{block}{Arrays}
    \begin{lstlisting}[language=Java]
Pixels[] array = new Pixels[5];
\end{lstlisting}
    \begin{itemize}
      \pause
    \item Arrays have intrinsic size
      \pause
    \item Arrays are safe to index out of bounds (throws exception)
    \end{itemize}
  \end{block}
\end{frame}

\begin{frame}[fragile]
  \frametitle{C++ types}
  \begin{block}{Primitive types}
    \begin{lstlisting}
uint32_t i;
double d;
\end{lstlisting}
    \begin{itemize}
      \pause
    \item Includes unsigned integer types
      \pause
    \item Integer types of defined sizes
    \end{itemize}
  \end{block}
\end{frame}

\begin{frame}[fragile]
  \frametitle{C++ types}
  \begin{block}{Classes}
  \begin{lstlisting}
Pixels         pixels;
§\pause§
Pixels        *pixelsptr1 = new Pixels();§\pause§
const Pixels  *pixelsptr2 = &pixels;
§\pause§
Pixels&        pixelsref(pixels);
\end{lstlisting}
  \begin{itemize}
    \pause
  \item Classes have no common root
    \pause
  \item All types may be instances, pointers or references
    \pause
  \item Pointers and references may refer to \texttt{const} type
    \pause
  \item Pointers may be \texttt{const}
    \pause
  \item Destruction is deterministic
  \end{itemize}
  \end{block}
\end{frame}

\begin{frame}[fragile]
  \frametitle{C++ types}
  \begin{block}{Arrays}
  \begin{lstlisting}
Pixels array[5];
\end{lstlisting}
  \begin{itemize}
    \pause
  \item Arrays “decay” to bare pointers
    \pause
  \item Arrays are not safe to index out of bounds
    \pause
  \item Size information lost at runtime
  \end{itemize}
  \end{block}
\end{frame}

\subsection{Reference handling}

\begin{frame}[fragile]
  \frametitle{C++ pointers: pitfalls of “dumb” pointers}
\begin{block}{Manual memory allocation}
  \begin{lstlisting}
{
  Image *i = new Image(filename);

  i->read_plane();

  // Memory not freed when pointer i goes out of scope
}
\end{lstlisting}
  \begin{itemize}
    \pause
  \item \texttt{new} and \texttt{delete} must always be paired
  \end{itemize}
\end{block}
\vbox to 0.4\textheight{%
}%
\end{frame}

\begin{frame}[fragile]
  \frametitle{C++ pointers: pitfalls of “dumb” pointers}
\begin{block}{Manual memory allocation and deallocation}
  \begin{lstlisting}
{
  Image *i = new Image(filename);

  i->read_plane();

  delete i;
}
\end{lstlisting}
  \begin{itemize}
  \item \texttt{new} and \texttt{delete} must always be paired
  \item Requires manual management of ownership and lifetime
    \pause
  \item This isn't sufficient
  \end{itemize}
\end{block}
\vbox to 0.4\textheight{%
}%
\end{frame}

\begin{frame}[fragile]
\begin{block}{Manual memory allocation and deallocation}
  \frametitle{C++ pointers: pitfalls of “dumb” pointers}
  \begin{lstlisting}
{
  Image *i = new Image(filename);

  i->read_plane(); // throws exception; memory leaked

  delete i; // never called
}
\end{lstlisting}
  \begin{itemize}
  \item \texttt{new} and \texttt{delete} must always be paired
  \item Requires manual management of ownership and lifetime
  \item Bare pointers are not exception-safe
    \pause
  \item Need to clean up for every exit point in a function
  \end{itemize}
\end{block}
\vbox to 0.4\textheight{%
}%
\end{frame}

\begin{frame}[fragile]
  \frametitle{C++ pointers: pitfalls of “dumb” pointers}
  \begin{block}{One correct solution}
  \begin{lstlisting}

{
  Image *i = new Image(filename);

  try {
    i->read_plane(); // throws exception
  } catch (const std::runtime_error& e) {
    delete i; // clean up
    throw; // rethrow
  }

  delete i; // never called for exceptions
}
\end{lstlisting}
  \begin{itemize}
    \pause
  \item Painful and error prone over an entire codebase
  \end{itemize}
  \end{block}
\end{frame}

\begin{frame}[fragile]
  \frametitle{C++ pointers: \texttt{std::shared\_ptr} as a “smart” pointer}
  \begin{lstlisting}
{ // Image *i = new Image(filename);
  std::shared_ptr<Image> i
    (std::make_shared<Image>(filename));
§\pause§
  i->read_plane(); // throws exception
§\pause§
  // Memory freed when i's destructor is run
}
\end{lstlisting}
  \begin{itemize}
    \item Memory is freed by the \texttt{shared\_ptr} destructor
      \pause
    \item \texttt{shared\_ptr} object lifetime manages the resource
      \pause
    \item May be used as class members; lifetime of class instance
      \pause
    \item Clean up for all exit points is automatic and safe
      \pause
    \item Allows ownership transfer and sharing
      \pause
    \item Allows reference without ownership using \texttt{weak\_ptr}
  \end{itemize}
\end{frame}

\begin{frame}[fragile]
  \frametitle{RAII: Resource Acquisition Is Initialisation}
    \begin{itemize}
      \pause
    \item Class is a proxy for a resource
      \pause
    \item Resource is acquired when object is intialised
    \item Resource is released when object is destroyed
      \pause
    \item Manage any resource (memory, files, locks, mutexes)
      \pause
    \item C++ language and runtime guarantees make resource management
      deterministic and reliable
      \pause
    \item Safe for use in any scope
      \pause
    \item Exception safe
      \pause
    \item Used throughout modern C++ libraries and applications
    \end{itemize}
\end{frame}

\begin{frame}[fragile]
  \frametitle{C++ reference variants}
  \begin{lstlisting}[basicstyle=\tiny\ttfamily]
//          Non-constant                              Constant
// -----------------------------       ---------------------------------------
// Pointer
                       Image *i;§\pause§                               const Image *i;
§\pause§                Image * const i;§\pause§                        const Image * const i;
§\pause§
// Reference
                       Image& i;                               const Image& i;
§\pause§
// Shared pointer
       std::shared_ptr<Image> i;               std::shared_ptr<const Image> i;
 const std::shared_ptr<Image> i;         const std::shared_ptr<const Image> i;
§\pause§
// Shared pointer reference
      std::shared_ptr<Image>& i;              std::shared_ptr<const Image>& i;
const std::shared_ptr<Image>& i;        const std::shared_ptr<const Image>& i;
§\pause§
// Weak pointer
         std::weak_ptr<Image> i;                 std::weak_ptr<const Image> i;
   const std::weak_ptr<Image> i;           const std::weak_ptr<const Image> i;
§\pause§
// Weak pointer reference
        std::weak_ptr<Image>& i;                std::weak_ptr<const Image>& i;
  const std::weak_ptr<Image>& i;          const std::weak_ptr<const Image>& i;
\end{lstlisting}
\pause
\begin{center}
  Java has one reference type.  Here, we have \textbf{22}…
\end{center}
\end{frame}

\begin{frame}[fragile]
  \frametitle{C++ reference usage rationalised}
\begin{block}{Class members}
  \begin{lstlisting}[basicstyle=\scriptsize\ttfamily]
Image i;                     // Concrete instance§\pause§
std::shared_ptr<Image> i;    // Reference§\pause§
std::weak_ptr<Image> i;      // Weak reference
\end{lstlisting}
\pause
\end{block}
\begin{block}{Arguments}
  \begin{lstlisting}[basicstyle=\scriptsize\ttfamily]
void read_plane(const Image& image); // ownership retained§\pause§
// ownership shared/transferred
void read_plane(const std::shared_ptr<Image>& image);
\end{lstlisting}
\pause
\end{block}
\begin{block}{Return}
  \begin{lstlisting}[basicstyle=\scriptsize\ttfamily]
                  Image get_image(); // ownership transferred§\pause§
                 Image& get_image(); // ownership retained§\pause§
 std::shared_ptr<Image> get_image(); // ownership shared/trans§\pause§
std::shared_ptr<Image>& get_image(); // ownership shared
\end{lstlisting}
\end{block}
\end{frame}

\begin{frame}[fragile]
  \frametitle{C++ reference usage rationalised}
  \begin{itemize}
  \item<1-> Safety: References can not be \texttt{null}.
  \item<2-> Storing polymorphic types requires use of a
    \texttt{shared\_ptr}.
  \item<3-> Safety: To avoid cyclic dependencies, use \texttt{weak\_ptr}.
  \item<4-> Safety: To allow object destruction while maintaining a safe
    reference, use \texttt{weak\_ptr}.
  \item<5-> \texttt{weak\_ptr} is not directly usable.
  \item<5-> \texttt{weak\_ptr} is convertible back to \texttt{shared\_ptr}
    for use \emph{if the object is still in existence}.
  \item<6-> C++11 \emph{move semantics} (\texttt{\&\&}) improve
    performance of ownership transfer
  \end{itemize}
\end{frame}

\subsection{Additional API concerns}

\begin{frame}[fragile]
  \frametitle{Safe array passing: \texttt{std::array} / \texttt{boost::array}}
  \begin{block}{C++ array problems}
  \begin{lstlisting}
class Image
{
  // Unsafe; size unknown
  uint8_t[] getLUT();
       void setLUT(uint8_t[]& lut);
};
\end{lstlisting}
  \begin{itemize}
  \item C++ arrays “decay” to “bare” pointers
  \item Pointers have no associated size information
  \end{itemize}
  \end{block}
\end{frame}

\begin{frame}[fragile]
  \frametitle{Safe array passing: \texttt{std::array} / \texttt{boost::array}}
  \begin{block}{\texttt{std::array}}
  \begin{lstlisting}
class Image
{
  // Safe; size defined
  typedef std::array<uint8_t, 256> LUT;
  const LUT& getLUT() const;
        void setLUT(const LUT&);
};
\end{lstlisting}
  \begin{itemize}
  \item<1-> \texttt{std::array} is an array-like object
  \item<2-> \texttt{std::array} size defined in the template
  \item<3-> \texttt{std::array} can be passed like any object
  \item<4-> Bounds checking with \texttt{.at()}
  \item<4-> Unchecked access with \texttt{[]}
  \end{itemize}
  \end{block}
\end{frame}

\begin{frame}[fragile]
  \frametitle{Containers storing different types}
\begin{block}{Types with a common base}
  \begin{lstlisting}
    std::vector<std::shared_ptr<Base> > v;
§\pause§    v.push_back(std::make_shared<Derived>());
\end{lstlisting}
  \begin{itemize}
    \pause
  \item Store any type derived from \texttt{Base}
  \end{itemize}
\end{block}
\pause
\begin{block}{Java containers can be problematic}
  \begin{itemize}
  \item Java can store root \texttt{Object} in containers
  \item Java can pass and return root \texttt{Object} in methods.
    \pause
  \item This isn't possible in C++: there is no root object.
  \item An alternative approach is needed.
  \end{itemize}
\end{block}
\end{frame}

\begin{frame}[fragile]
  \frametitle{Containers storing different types}
\begin{block}{Types with no common base: \texttt{boost::variant}}
  \begin{lstlisting}
    typedef boost::variant<int, std::string> variants;
§\pause§    std::vector<variants> v;
§\pause§    v.push_back(43);
§\pause§    v.push_back("ATTO 647N")
\end{lstlisting}
  \begin{itemize}
    \pause
  \item Store a set of discriminated types
    \pause
  \item Used to store original metadata
  \end{itemize}
\end{block}
\begin{block}{Arbitrary types: \texttt{boost::any}}
  \begin{lstlisting}
    std::vector<boost::any> v;
§\pause§    v.push_back(Anything);
\end{lstlisting}
  \begin{itemize}
    \pause
  \item Flexible, but need to cast to each type used to extract.
  \end{itemize}
\end{block}
\end{frame}

\begin{frame}[fragile]
  \frametitle{Interfaces}
\begin{block}{Java: Single inheritance, plus interfaces}
\end{block}
\begin{block}{C++: Multiple inheritance}
  \begin{itemize}
  \item Interfaces are classes with:
    \begin{itemize}
      \item No instance variables
      \item Pure virtual methods
      \item \texttt{protected} default constructor
      \item \texttt{public virtual} destructor
      \item Deleted copy constructor and assignment operator
    \end{itemize}
  \item Classes implementing interfaces:
    \begin{itemize}
      \item Use \texttt{public} inheritance for parent class
      \item Use \texttt{virtual public} inheritance for implemented interfaces
      \item \texttt{virtual} destructor
    \end{itemize}
  \end{itemize}
\end{block}
\end{frame}

\section*{Acknowledgements}

\frame{
  \frametitle{Acknowledgements}
  \parbox[t]{0.45\textwidth}{
    \begin{itemize}
    \item OME Team, Dundee
      \begin{itemize}
      \item Jason Swedlow
      \item Jean-Marie Burel
      \item Mark Carroll
      \item Andrew Patterson
      \item …and the rest of the team
      \end{itemize}
    \end{itemize}
  }
  \parbox[t]{0.45\textwidth}{
    \begin{itemize}
    \item Micron, Oxford
      \begin{itemize}
      \item Douglas Russell
      \end{itemize}
    \item Glencoe Software
      \begin{itemize}
      \item Melissa Linkert
      \item Josh Moore
      \end{itemize}
    \end{itemize}
  }

  \begin{center}
    \vcenteredhbox{\includegraphics[width=0.25\textwidth]{ome}} \hfill
    \vcenteredhbox{\includegraphics[width=0.2\textwidth]{dundee}}\hfill
    \vcenteredhbox{\includegraphics[width=0.5\textwidth]{wellcome}}
  \end{center}
}

\end{document}
